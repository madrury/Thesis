\documentclass[11pt]{amsart}
\usepackage{epsfig,amsmath}

\newtheorem{thm}[subsection]{Theorem}
\newtheorem{lem}[subsection]{Lemma}
\newtheorem{cor}[subsection]{Corollary}
\newtheorem{prop}[subsection]{Proposition}
\newtheorem{obs}[subsection]{Observation}

\def \la { \left\langle }
\def \ra { \right\rangle }
\def \C{ \mathbb{C} }
\def \ric{ \text{ric} }
\def \dim{ \text{dim} }
\def \Hom{ \text{Hom} }
\def \nablaamb{ \tilde{\nabla} }
\def \TMp{ TM^{+} }
\def \TMm{ TM^{-} }
\def \NMp{ NM^+ }
\def \NMm{ NM^- }
\def \TMps{ T^{*}M^{+} }

\theoremstyle{definition}
\newtheorem{definition}[subsection]{Definition}
\newtheorem{remark}[subsection]{Remark}

\begin{document}

\parskip 6pt
\parindent 0pt
\baselineskip 14pt

\section{Chern's Theorem}

We now turn to a more general situation.  Let $M$ be any complex manifold, holomorphically immersed in the space form $M_c$ so that the induced Kahler metric on $M$ is Einstein, with Einstein constant $\lambda$.  Suppose the dimension of $M$ is $n$, and the dimension of $M_c$ is $n+m$.  Let $\{ e_1, \ldots, e_{n+m} \}$ be a local unitary frame field for $TM_c^{+}$, chosen so that, when restricted to $M$, $\{ e_1, \ldots, e_n \}$ is a local unitary frame field for $\TMp$, and $\{ e_{n+1}, \ldots, e_{n+m} \}$ is a local unitary frame field for $\NMp$.  Let $\{ e^1, \ldots, e^{n+m} \}$ be the dual unitary frame field.  Again, we study the connection forms $\Theta$ and $\theta$ for $M_c$ and $M$.

\begin{remark}  We make use of index conventions similar to those in the previous section.  Lower case Greek indices $\alpha, \beta, \gamma$ will have the range $1,2, \ldots, n+m$, lower case Roman indices $i,j,k,l,r,s,t$ will have the range $1,2, \ldots, n$, and the lower case Roman indices $a,b,c$ will have the range $n+1, n+2, \ldots, n+m$.  Note that repeated indices in a single term are automatically summed over their ranges, though we will occasionally drop this convention for the sake of clarity when convenient.

Furthermore, every differential form in the subsequent calculation is always restricted to a form on $M$. In this context we have $e^a = 0$ for all $a$, and $\Theta^i_j = \theta^i_j$.

Finally, since we are using a unitary frame field, the curvature forms satisfy the skew-hermitian properties $\bar{\Theta}^{\alpha}_{\beta} = - \Theta^{\beta}_{\alpha}$, and $\bar{\theta}^i_j = - \theta^j_i$.
\end{remark}

We begin by differentiating the equation $e^a = 0$ and using the torsion free property of the metric connection.  The result is:
%
\begin{align*}
0 = d e^a = - \Theta^a_{\alpha} \wedge e^{\alpha} = - \Theta^a_i \wedge e^i
\end{align*}
%
It follows that $\Theta^a_i$ is a $(1,0)$ form on $M$, so there are some locally defined functions $A^a_{ij}$ so that:
%
\begin{align}
\Theta^a_i = A^a_{ij} e^j
\end{align}

\begin{lem} The second fundamental form of the immersion is locally expressed as:
%
\begin{align}
B( e_i, e_j ) = A^a_{ij} e_a
\end{align}
%
Consequently, the immersion is totally geodesic if and only if $A^a_{ij} = 0$ for all $a,i$ and $j$.
%
\end{lem}

\begin{proof} We simply calculate:
%
\begin{align*}
\la B(e_i, e_j), e_a \ra = \la \nablaamb_{e_i} e_j, e_a \ra = \la \Theta^{\alpha}_i (e_j) e_{\alpha}, e_a \ra = \Theta^a_i (e_j) = A^a_{ij}
\end{align*}
%
Recalling that $B(e_i, \bar{e}_j) = 0$, and $B(\bar{e}_i, \bar{e}_j ) = \overline{ B(e_i, e_j ) }$, we see that the lemma immediately follows.
%
\end{proof}

Since the second fundamental form is symmetric, an immediate consequence of the lemma above is that the symmetry condition $A^a_{ij} = A^a_{ji}$ holds for all $i,j$ and $a$.

Now we consider the curvature forms $\omega$ and $\Omega$ for $M$ and $M_c$ respectively, which satisfy the usual fundamental equations.  In particular:
%
$$\omega^i_i = d \theta^i_i + \theta^i_k \wedge \theta^k_i$$ 
%
and:
%
$$\Omega^i_i = d \Theta^i_i + \Theta^i_{\alpha} \wedge \Theta^{\alpha}_i = d \theta^i_i + \Theta^i_{\alpha} \wedge \Theta^{\alpha}_i$$
%
Hence:
%
\begin{align*}
\omega^i_i = \Omega^i_i - \Theta^i_a \wedge \Theta^a_i = \Omega^i_i - A^a_{ik} \bar{A}^a_{ij} e^k \wedge \bar{e}^j
\end{align*}
%
Now, since the metric on $M$ is Einstein, we have:
%
\begin{align*}
\omega^i_i = \frac{\lambda}{2} e^i \wedge \bar{e}^i
\end{align*}
%
And since $M_c$ has constant holomorphic curvature $c$, we have:
%
\begin{align*}
\Omega^i_i = \frac{(n+1)c}{4} e^i \wedge \bar{e}^i
\end{align*}
%
Substituting these expressions in above yields:
%
\begin{align*}
\frac{\lambda}{2} e^i \wedge \bar{e}^i = \frac{(n+1)c}{4} e^i \wedge \bar{e}^i - A^a_{ik} \bar{A}^a_{ij} e^k \wedge \bar{e}^j
\end{align*}
%
So:
%
\begin{align*}
A^a_{ik} \bar{A}^a_{ij} e^k \wedge \bar{e}^j = \left( \frac{(n+1)c}{4} - \frac{\lambda}{2} \right) e^i \wedge \bar{e}^i
\end{align*}
%
Therefore, since the forms $e^k \wedge \bar{e}^j$ form a basis for the space of $(1,1)$ forms on $M$, we conclude that:
%
\begin{align}
A^a_{ij} \bar{A}^a_{ik} = \delta^j_k \left( \frac{(n+1)c}{4} - \frac{\lambda}{2} \right)
\end{align}

\begin{cor}  The Einstein constant satisfies $\lambda \leq \frac{(n+1)c}{2}$, with equality if and only if $M$ is totally geodesic.  Furthermore, if $\lambda < \frac{(n+1)c}{2}$, then the linear map $\TMp \rightarrow \Hom ( \TMp, \NMp )$ defined by:
%
\begin{align*}
X \mapsto ( Y \mapsto B(X, Y) )
\end{align*}
%
is an injection.
%
\end{cor}

\begin{proof} Using equation (10) with $j=k$ we get:
%
\begin{align*}
0 \leq A^a_{ij} \bar{A}^a_{ij} = n \left( \frac{(n+1)c}{4} - \frac{\lambda}{2} \right)
\end{align*}
%
So it must be the case that $\lambda \leq \frac{(n+1)c}{2}$.  Equality holds only when $A^a_{ij} = 0$ for all $a,i,j$, which by the previous lemma, is equivalent to $M$ being totally geodesic.

Now, assume that $\lambda < \frac{(n+1)c}{2}$, and let $X = x_i e_i$ be a non-zero vector in $\TMp$.  We need to see that the linear map $Y \mapsto B(X,Y)$ is non-zero, so we calculate:
%
\begin{align*}
\la B(X, e_i), B(X, e_i) \ra &= x_j \bar{x}_k \la B(e_i, e_j), B(e_i, e_k) \ra = x_j \bar{x}_k \delta^a_b A^a_{ij} \bar{A}^b_{ik} \\
&= x_j \bar{x}_k A^a_{ij} \bar{A}^a_{jk} = x_j \bar{x}_k \delta^j_k \left( \frac{(n+1)c}{4} - \frac{\lambda}{2} \right) \\
&= x_j \bar{x}_j \left( \frac{(n+1)c}{4} - \frac{\lambda}{2} \right)
\end{align*}
%
Which is positive when $\lambda < \frac{(n+1)c}{2}$.  Therefore, there must be some $j$ so that $B(X,e_j) \neq 0$, which completes the proof of the corollary.
%
\end{proof}

To continue, we differentiate equation (8):
%
\begin{align*}
d \Theta^a_i = d A^a_{ik} \wedge e^k + A^a_{ij} d e^j
\end{align*}
%
The fundamental equation gives:
%
\begin{align*}
d \Theta^a_i = \Omega^a_i - \Theta^a_{\alpha} \wedge \Theta^{\alpha}_i
\end{align*}
%
And using the curvature formula for space forms, we get:
%
\begin{align*}
\Omega^a_i = \frac{c}{4} e^a \wedge \bar{e}^i = 0
\end{align*}
%
So:
%
\begin{align*}
d \Theta^a_i &= - \Theta^a_{\alpha} \wedge \Theta^{\alpha}_i = - \Theta^a_j \wedge \Theta^j_i - \Theta^a_b \wedge \Theta^b_i = A^a_{jk} \theta^j_i \wedge e^k - A^b_{ik} \Theta^a_b \wedge e^k
\end{align*}
%
On the other hand, the torsion free metric property gives:
%
\begin{align*}
A^a_{ij} d e^j = - A^a_{ij} \theta^j_k \wedge e^k
\end{align*}
%
Substituting all these ingredients in now yields:
%
\begin{align*}
0 = \left( A^a_{jk} \theta^j_i - A^b_{ik} \Theta^a_b - d A^a_{ik} + A^a_{ij} \theta^j_k \right) \wedge e^k
\end{align*}
%
Therefore, the form appearing as a coefficient of $e^k$ must be a $(1,0)$ form, and so there are some locally defined functions $B^a_{ijl}$ that satisfy:
%
\begin{align}
B^a_{ijl} e^l = d A^a_{ij} + A^b_{ij} \Theta^a_b - A^a_{kj} \theta^k_i - A^a_{ik} \theta^k_j
\end{align}

\begin{lem}  The first covariant derivative of the second fundamental form has the local expression:
%
\begin{align*}
\nabla B (e_i, e_j, e_l) = B^a_{ijl} e_a
\end{align*}
%
Consequently, the second fundamental form is parellel if and only if $B^a_{ijl} = 0$ for all $a,i,j$ and $l$.
%
\end{lem}

\begin{proof}  We use the definition of the covariant derivative to calculate:
%
\begin{align*}
\nabla B (e_i, e_j, e_l) &= ( \nabla_{e_l} B )(e_i, e_j) \\
&= \nabla^{\perp}_{e_l} \left( B( e_i, e_j) \right) - B( \nabla_{e_l} e_i, e_j ) - B( e_i, \nabla_{e_j} e_j ) \\
&= \nabla^{\perp}_{e_l} \left( A^a_{ij} e_a \right) - B( \theta^k_i (e_l) e_k, e_j ) - B( e_i, \theta^k_j (e_j) e_k ) \\
&= \left( d A^a_{ij} + A^b_{ij} \Theta^a_b - A^a_{kj} \theta^k_i - A^a_{ik} \theta^k_j \right)(e_l) e_a \\
&= B^a_{ijl} e_a
\end{align*}
%
which completes the proof.
%
\end{proof}

\begin{cor}  The numbers $B^i_{ijl}$ are invariant with regards to all permutations of the lower three indicies.  That is, $B^a_{ijl} = B^a_{jil} = B^a_{ilj}$. 
\end{cor}  
%
\begin{proof} The relation $B^a_{ijl} = B^a_{jil}$ is an easy consequence of the symmetry of $A^a_{ij}$ and equation (11). 

Verifying the other symmetry relation is considerably more subtle.  By the previous lemma, it suffices to check the tensor equation:
%
\begin{align*}
\nabla B (e_i, e_j, e_l) - \nabla B (e_i, e_l, e_j ) = 0
\end{align*}
%
Fix a general point $p$ in $M$, and for the moment assume (for convenience) that the frame $\{ e_1, e_2, \ldots, e_{n+m} \}$ is parallel at $p$.  The following calculations are always evaluated at the point $p$.  We have:
%
\begin{align*}
\nabla B (e_i, e_j, e_l) - \nabla B (e_i, e_l, e_j ) &= \nabla^{\perp}_{e_l}( B(e_i, e_j) ) - \nabla^{\perp}_{e_j}( B(e_i, e_l) )\\
&= d A^a_{ij} (e_l) e_a - d A^a_{il} (e_j) e_a
\end{align*}
%
Now since:
%
\begin{align*}
A^a_{ij} = \Theta^a_i (e_j) = e^a( \nabla_{e_j} e_i ) = \la \nabla_{e_j} e_i, e_a \ra
\end{align*}
%
compatability of the connection with the metric yields:
%
\begin{align*}
d A^a_{ij} (e_l) = e_l \la \nabla_{e_j} e_i, e_a \ra = \la \nabla_{e_l} \nabla_{e_j} e_i, e_a \ra
\end{align*}
%
where we have used that $\nabla_{e_l} e_{a} = 0$ at $p$.  Therefore:
%
\begin{align*}
d A^a_{ij} (e_l) - d A^a_{il} (e_j) &= \la \nabla_{e_l} \nabla_{e_j} e_i, e_a \ra - \la \nabla_{e_j} \nabla_{e_l} e_i, e_a \ra \\
&= \la R(e_l, e_j) e_i, e_a \ra \\
&= 0
\end{align*}
%
since $e_l$ and $e_j$ lie in $TM^{+}$.
%
\end{proof}

We now want to find an orthogonality relation between the $A^a_{ij}$ and the $B^a_{ijl}$.  To begin, we differentiate equation (10):
%
\begin{align*}
0 = \bar{A}^a_{ik} d A^a_{ij} + A^a_{ij} d \bar{A}^a_{ik}
\end{align*}
%
To simplify these two terms we use equation (11), the results are:
%
\begin{align*}
\bar{A}^a_{ik} d A^a_{ij} = \bar{A}^a_{ik} B^a_{ijl} e^l + \bar{A}^a_{ik} A^a_{il} \theta^l_j + \bar{A}^a_{ik} A^a_{lj} \theta^l_i - \bar{A}^a_{ik} A^b_{ij} \Theta^a_b
\end{align*}
%
and:
%
\begin{align*}
A^a_{ij} d \bar{A}^a_{ik} = A^a_{ij} \bar{B}^a_{ikl} \bar{e}^l - A^a_{ij} \bar{A}^a_{lk} \theta^i_l - A^a_{ij} \bar{A}^a_{il} \theta^k_l + A^a_{ij} \bar{A}^b_{ik} \Theta^b_a
\end{align*}
%
Substituting in these two formula, we see, upon re-indexing, that two pairs of terms cancel, and we are left with:
%
\begin{align*}
0 = \bar{A}^a_{ik} B^a_{ijl} e^l + A^a_{ij} \bar{B}^a_{ikl} \bar{e}^l + \bar{A}^a_{ik} A^a_{il} \theta^l_j - A^a_{ij} \bar{A}^a_{il} \theta^k_l
\end{align*}
%
To deal with the last two terms here, we call upon formula (10), which yields:
%
\begin{align*}
\bar{A}^a_{ik} A^a_{il} \theta^l_j - A^a_{ij} \bar{A}^a_{il} \theta^k_l &= \delta^k_l \left( \frac{(n+1)c}{4} - \frac{\lambda}{2} \right) \theta^l_j - \delta^j_l \left( \frac{(n+1)c}{4} - \frac{\lambda}{2} \right) \theta^k_l \\
&= \left( \frac{(n+1)c}{4} - \frac{\lambda}{2} \right) \left( \theta^k_j - \theta^k_j \right) \\
&= 0
\end{align*}
%
Therefore:
%
\begin{align*}
0 = \bar{A}^a_{ik} B^a_{ijl} e^l + A^a_{ij} \bar{B}^a_{ikl} \bar{e}^l
\end{align*}
%
and we conclude the orthogonality relation:
%
\begin{align}
\bar{A}^a_{ik} B^a_{ijl} = 0
\end{align}
%
For all $k,j$ and $l$.

For the next corollary, we denote by $\TMp \odot \TMp$ the space of symmetric 2-tensors on $\TMp$.  

\begin{cor} The image of the linear map $\TMp \odot \TMp \rightarrow \Hom( \TMp, \NMp )$ defined by:
%
\begin{align*}
X \odot Y \mapsto ( Z \mapsto( \nabla B )( X, Y, Z ) )
\end{align*}
%
is orthogonal to the image of the map $\TMp \rightarrow \Hom( \TMp, \NMp )$. 
%
\end{cor}

\begin{proof}  Let $X = x_j e_j$ and $Y = y_{rs} e_r \odot e_s$.  Then we have:
%
\begin{align*}
\la B( X, e_i ), (\nabla_{e_i} B) ( Y ) \ra &= x_j \bar{y}_{rs} \la B(e_i, e_j), ( \nabla_{e_i} B )( e_r, e_s ) \ra \\
&= x_j \bar{y}_{rs} \delta^a_b A^a_{ij} \bar{B}^b_{irs} = x_j \bar{y}_{rs} A^a_{ij} \bar{B}^a_{irs} \\
&= 0
\end{align*}
%
Which completes the proof.
%
\end{proof}

We now want to get an expression for $B^a_{ijl} \bar{B}^a_{ijr}$.  To do so, we begin by differentiating the equation $\bar{A}^a_{ik} B^a_{ijl} e^l = 0$.  The result is:
%
\begin{align*}
0 = B^a_{ijl} d \bar{A}^a_{ik} \wedge e^l + \bar{A}^a_{ik} d( B^a_{ijl} e^l ) 
\end{align*}
%
Using equation (11) to substitute into the first term, and making use of the orthogonality relation (12), we get:
%
\begin{align*}
B^a_{ijl} d \bar{A}^a_{ik} \wedge e^l= B^a_{ijl} \bar{B}^a_{ikr} \bar{e}^r \wedge e^l + \bar{A}^b_{ik} B^a_{ikl} \Theta^b_a \wedge e^l - \bar{A}^a_{rk} B^a_{ijl}  \theta^i_r \wedge e^l
\end{align*}
%
For the second term, we differentiate equation (11) to get:
%
\begin{align*}
\bar{A}^a_{ik} d( B^a_{ijl} e^l ) =& \bar{A}^a_{ik} d A^b_{ij} \wedge \Theta^a_b + \bar{A}^a_{ik} A^b_{ij} d \Theta^a_b - \bar{A}^a_{ik} d
A^a_{rj} \wedge \theta^r_i \\
&- \bar{A}^a_{ik} A^a_{rj} d \theta^r_i - \bar{A}^a_{ik} d A^a_{ir} \wedge \theta^r_j - \bar{A}^a_{ik} A^a_{ir} d \theta^r_j
\end{align*}
%
We simplify these terms one by one, using always the orthogonality relations (11), (12), and the following remark:

\begin{remark}  We take a look at the $d \Theta$ and $d \theta$ terms.  For instance, to calculate $d \Theta^a_b$, we apply the fundamental equation, and the formula for the curvature in a space form to get:
%
\begin{align*}
d \Theta^a_b &= \Omega^a_b - \Theta^a_{\alpha} \wedge \Theta^{\alpha}_b \\
&= \frac{c}{4}( \delta^a_b e^l \wedge \bar{e}^l + e^a \wedge \bar{e}^b ) - \Theta^a_l \wedge \Theta^l_b - \Theta^a_c \wedge \Theta^c_b \\
&= \frac{c}{4} \delta^a_b e^l \wedge \bar{e}^l + A^a_{lr} \bar{A}^b_{ls} e^r \wedge \bar{e}^s - \Theta^a_c \wedge \Theta^c_b
\end{align*}
%
Similar analysis yields:
%
\begin{align*}
d \theta^r_i = \frac{c}{4} \delta^r_i e^l \wedge \bar{e}^l + \frac{c}{4} e^r \wedge \bar{e}^i - \theta^r_l \wedge \theta^l_i + \bar{A}^b_{rl} A^b_{is} \bar{e}^l \wedge e^s
\end{align*}
%
and:
%
\begin{align*}
d \theta^r_j = \frac{c}{4} \delta^r_j e^l \wedge \bar{e}^l + \frac{c}{4} e^r \wedge \bar{e}^j - \theta^r_l \wedge \theta^l_j + \bar{A}^b_{rl} A^b_{js} \bar{e}^l \wedge e^s
\end{align*}
%
\end{remark}

\bigskip

Now, carefully simplifying each of the six terms yields:
%
\begin{align*}
\bar{A}^a_{ik} d A^b_{ij} \wedge \Theta^a_b =& B^b_{ijl} \bar{A}^a_{ik} e^l \wedge \Theta^a_b - \bar{A}^a_{ik} A^c_{ij} \Theta^b_c \wedge \Theta^a_b\\
&+ \bar{A}^a_{ik} A^b_{rj} \theta^r_i \wedge \Theta^a_b + \bar{A}^a_{ik} A^b_{ir} \theta^r_j \wedge \Theta^a_b
\end{align*}
%
\begin{align*}
\bar{A}^a_{ik} A^b_{ij} d \Theta^a_b = \frac{c}{4} \bar{A}^a_{ik} A^a_{ij} e^l \wedge \bar{e}^l + \bar{A}^a_{ik} \bar{A}^b_{ls} A^a_{lr} A^b_{ij} e^r \wedge \bar{e}^s - \bar{A}^a_{ik} A^b_{ij} \Theta^a_c \wedge \Theta^c_b
\end{align*}
%
\begin{align*}
- \bar{A}^a_{ik} d A^a_{rj} \wedge \theta^r_i =& - \bar{A}^a_{ik} B^a_{rjl} e^l \wedge \theta^r_i + \bar{A}^a_{ik} A^b_{rj} \Theta^a_b \wedge \theta^r_i \\
&- \bar{A}^a_{ik} A^a_{lj} \theta^l_r \wedge \theta^r_i - \bar{A}^a_{ik} A^a_{rl} \theta^l_j \wedge \theta^r_i
\end{align*}
%
\begin{align*}
- \bar{A}^a_{ik} A^a_{rj} d \theta^r_i =& - \frac{c}{4} \bar{A}^a_{ik} A^a_{ij} e^l \wedge \bar{e}^l - \frac{c}{4} \bar{A}^a_{ik} A^a_{rj} e^r \wedge \bar{e}^i \\
&+ \bar{A}^a_{ik} A^a_{rj} \theta^r_l \wedge \theta^l_i - \bar{A}^a_{ik} \bar{A}^b_{rl} A^a_{rj} A^b_{is} \bar{e}^l \wedge e^s
\end{align*}
%
\begin{align*}
- \bar{A}^a_{ik} d A^a_{ir} \wedge \theta^r_j = \bar{A}^a_{ik} A^b_{ir} \Theta^a_b \wedge \theta^r_j - \bar{A}^a_{ik} A^a_{lr} \theta^l_i \wedge \theta^r_j - \bar{A}^a_{ik} A^a_{il} \theta^l_r \wedge \theta^r_j
\end{align*}
%
\begin{align*}
- \bar{A}^a_{ik} A^a_{ir} d \theta^r_j =& - \frac{c}{4} \bar{A}^a_{ik} A^a_{ij} e^l \wedge \bar{e}^l - \frac{c}{4} \bar{A}^a_{ik} A^a_{ir} e^r \wedge \bar{e}^j  \\
&+ \bar{A}^a_{ik} A^a_{ir} \theta^r_l \wedge \theta^l_j - \bar{A}^a_{ik} \bar{A}^b_{rl} A^a_{ir} A^b_{js} \bar{e}^l \wedge e^s
\end{align*}

Substituting in all these results, we see that all the terms containing either a $\theta$ or a $\Theta$ cancel, and we are left with:
%
\begin{align*}
0 =& B^a_{ijl} \bar{B}^a_{ikr} \bar{e}^e \wedge e^l - \frac{c}{4} \bar{A}^a_{ik} A^a_{ij} e^l \wedge \bar{e}^l - \frac{c}{4} \bar{A}^a_{ik} A^a_{rj} e^r \wedge \bar{e}^i - \frac{c}{4} \bar{A}^a_{ik} A^a_{ir} e^r \wedge \bar{e}^j \\
&+ \bar{A}^a_{ik} \bar{A}^b_{ls} A^a_{lr} A^b_{ij} e^r \wedge \bar{e}^s + \bar{A}^a_{ik} \bar{A}^b_{rl} A^a_{ij} A^b_{is}  e^s \wedge \bar{e}^l +
\bar{A}^a_{ik} \bar{A}^b_{rl} A^a_{ir} A^b_{js} e^s \wedge \bar{e}^l
\end{align*}
%
We can simplify three of these terms further by applying formula (10):
%
\begin{align*}
 - \frac{c}{4} \bar{A}^a_{ik} A^a_{ij} e^l \wedge \bar{e}^l &= - \frac{c}{4} \delta^j_k \left( \frac{(n+1)c}{4} - \frac{\lambda}{2} \right) e^l \wedge \bar{e}^l \\
&= - \frac{c}{4} \delta^j_k \delta^r_s \left( \frac{(n+1)c}{4} - \frac{\lambda}{2} \right) e^s \wedge \bar{e}^r
\end{align*}
%
\begin{align*}
- \frac{c}{4} \bar{A}^a_{ik} A^a_{ir} e^r \wedge \bar{e}^j &= - \frac{c}{4} \delta^k_r \left( \frac{(n+1)c}{4} - \frac{\lambda}{2} \right) e^r \wedge \bar{e}^j \\
&= - \frac{c}{4} \delta^k_r \delta^j_s \left( \frac{(n+1)c}{4} - \frac{\lambda}{2} \right) e^r \wedge \bar{e}^s \\
&= - \frac{c}{4} \delta^k_s \delta^j_r \left( \frac{(n+1)c}{4} - \frac{\lambda}{2} \right) e^s \wedge \bar{e}^r
\end{align*} 
%
\begin{align*}
\bar{A}^a_{ik} \bar{A}^b_{rl} A^a_{ir} A^b_{js} e^s \wedge \bar{e}^l  &= \delta^k_r \left( \frac{(n+1)c}{4} - \frac{\lambda}{2} \right)  \bar{A}^b_{rl} \bar{A}^b_{js} e^s \wedge \bar{e}^l \\
&= \left( \frac{(n+1)c}{4} - \frac{\lambda}{2} \right)  \bar{A}^a_{rk} \bar{A}^a_{js} e^s \wedge \bar{e}^r
\end{align*}
%
We get, upon gathering like terms and changing around some of the summation indices:
%
\begin{align*}
B^a_{ijs} \bar{B}^a_{ikr} e^s \wedge \bar{e}^r =& - \frac{c}{4} \left( \frac{(n+1)c}{4} - \frac{\lambda}{2} \right) \left( \delta^j_k \delta^r_s + \delta^k_s \delta^j_r \right) e^s \wedge \bar{e}^r \\
&+ \left( \frac{nc}{4} - \frac{\lambda}{2} \right) \bar{A}^a_{rk} A^a_{sj} e^s \wedge \bar{e}^r + \bar{A}^a_{ik} \bar{A}^b_{lr} A^a_{ls} A^b_{ij} e^s \wedge \bar{e}^r \\
&+ \bar{A}^a_{ik} \bar{A}^b_{lr} A^a_{lj} A^b_{is} e^s \wedge \bar{e}^r
\end{align*}
%
Since the forms $e^s \wedge \bar{e}^r$ form a basis for the $(1,1)$ forms on $M$, we can isolate the coefficients in the equation above to finally get:
%
\begin{align}
B^a_{ijs} \bar{B}^a_{ikr} =& - \frac{c}{4} \left( \frac{(n+1)c}{4} - \frac{\lambda}{2} \right) \left( \delta^j_k \delta^r_s + \delta^k_s \delta^j_r \right) + \left( \frac{nc}{4} - \frac{\lambda}{2} \right)  \bar{A}^a_{rk} A^a_{sj}\\
&+ \bar{A}^a_{ik} \bar{A}^b_{lr} A^a_{ls} A^b_{ij} + \bar{A}^a_{ik} \bar{A}^b_{lr} A^a_{lj} A^b_{is}  \notag 
\end{align}

\begin{cor}  If $\lambda < \frac{(n+1)c}{2}$ then, in fact, $\lambda \leq \frac{nc}{2}$, with equality holding if and only if the second fundamental form $B$ is parallel.
\end{cor}
%
\begin{proof} For convienince, write $ K =\left( \frac{(n+1)c}{4} - \frac{\lambda}{2} \right)$.  Set $k = j$ and $r = s$ in formula (13) to get:
%
\begin{align*}
B^a_{ijs} \bar{B}^a_{ijs} =& -\frac{c}{4} K ( \delta^j_j \delta^r_r + \delta^r_j \delta^r_j ) + \left( K - \frac{c}{4} \right) \bar{A}^a_{sj} A^a_{sj} \\
&+ \bar{A}^a_{ij} \bar{A}^b_{sr} A^a_{sr} A^b_{ij} + \bar{A}^a_{ij} \bar{A}^b_{sr} A^a_{sj} A^b_{ir}
\end{align*}
%
It is easy to see that:
%
\begin{align*}
-\frac{c}{4} K ( \delta^j_j \delta^r_r + \delta^r_j \delta^r_j ) = -\frac{c}{4} K n (n+1)
\end{align*}
%
The second and fourth terms can be computed using formula (10):
%
\begin{align*}
\left( K - \frac{c}{4} \right) \bar{A}^a_{sj} A^a_{sj} = n K \left( K - \frac{c}{4} \right)
\end{align*}
%
\begin{align*}
\bar{A}^a_{ij} \bar{A}^b_{sr} A^a_{sj} A^b_{ir} = \delta^i_s \delta^i_s K^2 = n K^2
\end{align*}
%
The third term we can merely estimate, using the Cauchy-Swartz inequality.  For sake of clarity, in the estimate we temporarily drop our summation convention.  We have:
%
\begin{align*}
\sum_{abijsr} \bar{A}^a_{ij} \bar{A}^b_{sr} A^a_{sr} A^b_{ij} &= \sum_{ijsr} \left( \sum_{a} \bar{A}^a_{ij} A^a_{sr} \right)\left( \sum_{b} \bar{A}^b_{sr} A^b_{ij} \right) = \sum_{ijsr} \left| \sum_{a} \bar{A}^a_{ij} A^a_{sr} \right|^2 \\
&\leq \sum_{ijrs} \left| \sum_{a} \bar{A}^a_{ij} \right|^2 \left| \sum_{b}  A^b_{sr} \right|^2 \\
&= \sum_{ijrs} \left( \sum_{a} \bar{A}^a_{ij} A^a_{ij} \right) \left( \sum_{b}  A^b_{sr} \bar{A}^b_{sr} \right)\\
&= n^2 K^2
\end{align*}
%
Where we have used (10) in the last equality.  Putting this all together, and doing a little calculation, we get:
%
\begin{align*}
B^a_{ijs} \bar{B}^a_{ijs} &\leq -\frac{c}{4} K n(n+1) + n K \left( K - \frac{c}{4} \right) + n K^2 + n^2 K^2 \\
&= n(n+2) \left( \frac{(n+1)c}{4} - \frac{\lambda}{2} \right) \left( \frac{nc}{4} - \frac{\lambda}{2} \right)
\end{align*}
%
Now $B^a_{ijs} \bar{B}^a_{ijs}$ and $\left( \frac{(n+1)c}{4} - \frac{\lambda}{2} \right)$ are both non-negative, so it follows that $\left( \frac{nc}{4} - \frac{\lambda}{2} \right)$ must be as well.  That is, $\lambda \leq \frac{nc}{2}$.  Equality occurs if and only if $B^a_{ijs} = 0$ for every $a,i,j,s$, which by lemma 6, means that the second fundamental form $B$ is parallel.
\end{proof}
 
With the formula for $B^a_{ijk} \bar{B}^a_{irs}$ in hand we want to study the map $\TMp \odot \TMp \rightarrow \Hom( \TMp, \NMp )$ in more detail.  So let $X = x_{jk} e_j \odot e_k$ be a non-zero vector in $\TMp \odot \TMp$. Then:
%
\begin{align*}
\la \nabla B (X, e_i), \nabla B(X, e_i) \ra &=  x_{jk} \bar{x}_{rs} \la \nabla B(e_j, e_k, e_i), \nabla B(e_r, e_s, e_i) \ra \\
&= x_{jk} \bar{x}_{rs} B^a_{ijk} \bar{B}^a_{irs}
\end{align*}
%
Using (13) we get that this is equal to:
%
\begin{align*}
- \frac{c}{4} \left( \frac{(n+1)c}{4} - \frac{\lambda}{2} \right)\left( \delta^j_r \delta^s_k + \delta^k_r \delta^s_j \right) x_{jk} \bar{x}_{rs} + \left( \frac{nc}{4} - \frac{\lambda}{2} \right) A^b_{rs} \bar{A}^b_{jk} x_{jk} \bar{x}_{rs} \\
+ \bar{A}^b_{mr} \bar{A}^c_{ls} A^b_{lk} A^c_{mj} x_{jk} \bar{x}_{rs} + \bar{A}^b_{mr} \bar{A}^c_{ls} A^b_{lj} A^c_{mk} x_{jk} \bar{x}_{rs}
\end{align*}
%
Looking at the first two of these terms, and recalling that $x_{ij} = x_{ji}$, we see that:
%
\begin{align*}
- \frac{c}{4} \left( \frac{(n+1)c}{4} - \frac{\lambda}{2} \right) \left( \delta^j_r \delta^s_k + \delta^k_r \delta^s_j \right) x_{jk} \bar{x}_{rs} &= - \frac{c}{2} \left( \frac{(n+1)c}{4} - \frac{\lambda}{2} \right) x_{jk} \bar{x}_{jk} \\
&= - \frac{c}{2} \left( \frac{(n+1)c}{4} - \frac{\lambda}{2} \right) \sum_{jk} \left| x_{jk} \right|^2
\end{align*}
%
For the second term we observe that:
%
\begin{align*}
\left( \frac{nc}{4} - \frac{\lambda}{2} \right) A^b_{rs} \bar{A}^b_{jk} x_{jk} \bar{x}_{rs} = \left( \frac{nc}{4} - \frac{\lambda}{2} \right) \sum_{b} \left| A^b_{jk} x_{jk} \right|^2
\end{align*}
%
For the final two terms, we calculate:
%
\begin{align*}
\bar{A}^b_{mr} \bar{A}^c_{ls} & A^b_{lk} A^c_{mj} x_{jk} \bar{x}_{rs} + \bar{A}^b_{mr} \bar{A}^c_{ls} A^b_{lj} A^c_{mk} x_{jk} \bar{x}_{rs} \\ 
&= \bar{A}^b_{mr} \bar{A}^c_{ls} A^b_{lk} A^c_{mj} x_{jk} \bar{x}_{rs} + \bar{A}^c_{mr} \bar{A}^b_{ls} A^c_{lj} A^b_{mk} x_{jk} \bar{x}_{rs} \\
&=  \sum_{lmbc} \left( \sum_{rs} ( \bar{A}^b_{mr} \bar{A}^c_{ls} - \bar{A}^c_{mr} \bar{A}^b_{ls} ) \bar{x}_{rs} \right) \left( \sum_{jk} ( A^b_{lk} A^c_{mj} -  A^c_{lj} A^b_{mk} ) x_{jk} \right)\\
&\quad +  \bar{A}^b_{mr} \bar{A}^c_{ls} A^c_{lj} A^b_{mk} x_{jk} \bar{x}_{rs} + \bar{A}^c_{mr} \bar{A}^b_{ls} A^b_{lk} A^c_{mj} x_{jk} \bar{x}_{rs} \\
&= - \sum_{lmbc} \left| \sum_{jk} \left( A^b_{lk} A^c_{mj} -  A^c_{lj} A^b_{mk}  \right) x_{jk} \right|^2 + 2 \delta^r_k \delta^j_s \left( \frac{(n+1)c}{4} - \frac{\lambda}{2} \right)^2 x_{jk} \bar{x}_{rs} \\
&= - \sum_{lmbc} \left| \sum_{jk} \left( A^b_{lk} A^c_{mj} -  A^c_{lj} A^b_{mk} \right) x_{jk} \right|^2 + 2 \left( \frac{(n+1)c}{4} - \frac{\lambda}{2} \right)^2 \sum_{jk} \left| x_{jk} \right|^2
\end{align*}
%
Therefore, we have:
%
\begin{align*}
\la \nabla B (X, e_i), \nabla B(X, e_i) \ra &= - \sum_{lmbc} \left| \sum_{jk} \left( A^b_{lk} A^c_{mj} -  A^c_{lj} A^b_{mk} \right) x_{jk} \right|^2 \\
&+ \left( \frac{nc}{4} - \frac{\lambda}{2} \right) \sum_{b} \left| A^b_{jk} x_{jk} \right|^2 \\
&+ \left( \frac{(n+1)c}{4} - \frac{\lambda}{2} \right) \left( 2 \left( \frac{(n+1)c}{4} - \frac{\lambda}{2} \right) - \frac{c}{2} \right) \sum_{jk} \left| x_{jk} \right|^2 \\
&=  - \sum_{lmbc} \left| \sum_{jk} \left( A^b_{lk} A^c_{mj} -  A^c_{lj} A^b_{mk} \right) x_{jk} \right|^2 \\
&+ \left( \frac{nc}{4} - \frac{\lambda}{2} \right) \sum_{b} \left| A^b_{jk} x_{jk} \right|^2 \\
&+ 2 \left( \frac{(n+1)c}{4} - \frac{\lambda}{2} \right) \left( \frac{nc}{4} - \frac{\lambda}{2} \right) \sum_{jk} \left| x_{jk} \right|^2
\end{align*}
%
We now have all the pieces in hand that we need to prove Chern's theorem.

\begin{thm} 
Suppose that $m=1$, then the Einstein constant $\lambda$ is either $\frac{(n+1)c}{2}$ or $\frac{nc}{2}$.  In the first case, the immersion is totally geodesic, and in the second, the second fundamental form $B$ is parallel. 
\end{thm}

\begin{proof} When $m=1$ the first term in the previous formula is easily seen to vanish, so:
%
\begin{align*}
\la \nabla B (X, e_i), \nabla B(X, e_i) \ra &= \left( \frac{nc}{4} - \frac{\lambda}{2} \right) \sum_{b} \left| A^b_{jk} x_{jk} \right|^2 \\
&+ \left( \frac{(n+1)c}{4} - \frac{\lambda}{2} \right) \left( 2 \left( \frac{(n+1)c}{4} - \frac{\lambda}{2} \right) - \frac{c}{2} \right) \sum_{jk} \left| x_{jk} \right|^2 \\
\end{align*}
If $\lambda \neq \frac{(n+1)c}{2}, \frac{nc}{2}$, then, by a previous corollary, it must be the case that $\lambda < \frac{nc}{2}$, hence the right hand side is positive.  Therefore:
%
$$ \la \nabla B (X, e_i), \nabla B(X, e_i) \ra > 0 $$
%
and hence the map $\TMp \odot \TMp \rightarrow \Hom( \TMp, \NMp )$ is injective.  We know already that the map $TM^{+} \rightarrow \Hom( TM^{+}, NM^{+} )$ is injective, and that the images of these two maps are orthogonal.  Since the dimension of $TM^{+}$ is $n$, the dimension of $\TMp \odot \TMp$ is $\frac{n(n-1)}{2}$.  The dimension of $ \Hom ( \TMp, \NMp ) $ is $ n \times 1 = n$, so we must conclude that:
%
$$ n + \frac{n(n-1)}{2} < n $$
%
which is a contradiction unless $n=1$, a case we have already studied in the previous chapter.
\end{proof}

\end{document}