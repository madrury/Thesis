\documentclass[11pt]{amsart}
\usepackage{epsfig,amsmath}

\newtheorem{thm}[subsection]{Theorem}
\newtheorem{lem}[subsection]{Lemma}
\newtheorem{cor}[subsection]{Corollary}
\newtheorem{prop}[subsection]{Proposition}
\newtheorem{obs}[subsection]{Observation}


\theoremstyle{definition}
\newtheorem{definition}[subsection]{Definition}
\newtheorem{remark}[subsection]{Remark}

\def \CP{ \mathbb{C}P }
\def \C{ \mathbb{C} }
\def \CH{ \mathbb{C}H }
\def \Mamb{ \mathcal{M} }
\def \del{ \partial }
\def \delbar{ \bar{\partial} }
\def \Ric{ \text{Ric} }
\def \tr{ \text{tr} }
\def \ric{ \text{ric} }
\def \Rcurv{ \mathcal{R} }
\def \re{ \text{Re} }
\def \im{ \text{Im} }
\def \Re{ \text{Re} }
\def \Im{ \text{Im} }


\begin{document}

\parskip 6pt
\parindent 0pt
\baselineskip 14pt

\section{ Background on Kahler Geometry }

In this section we outline the basic properties of Kahler geometry that we will be using throughout this paper.

\subsection{Invariant Metrics and the Associated 2-form }

Let $M$ be a complex manifold with complex structure $J$.
%
\begin{definition} A Riemannian metric on $M$ is called $J$-invariant if $g(JX, JY) = g(X, Y)$ for all $X,Y \in TM$.
\end{definition}
%
Note that any complex manifold can be given a $J$-invariant Riemannian metric, for we can locally use the standard euclidean metric on $\C^n$, and then assemble one globally using a partition of unity.  
%
We denote also by $g$ the complex bilinear extension of the given metric to the complexified tangent bundle $T_{\C}M$.  

As usual we consider the local holomorphic coordinate frame fields:
%
$$ Z_i = \frac{ \del }{ \del z_i } = \frac{1}{2} \left( \frac{ \del }{ \del x_i } - i \frac{ \del }{ \del y_i } \right) = \frac{1}{2}(X_i - i Y_i) $$
%
and:
%
$$\bar{Z}_i = \frac{ \del }{ \del \bar{z}_i } = \frac{1}{2} \left( \frac{ \del }{ \del x_i } + i \frac{ \del }{ \del y_i } \right) = \frac{1}{2} ( X_i + i Y_i ) $$
%
Then, since  $g$ is $J$-invariant and $Y_i = J X_i$, we have:
%
\begin{align*}
g(Z_i, Z_j) &= g( X_i - i Y_i, X_j - i Y_j ) \\
&= g( X_i, X_j ) - i g(X_i, Y_j) - i g(Y_i, X_j) - g(Y_i, Y_j) \\
&= g( X_i, X_j ) - i g(X_i, Y_j) + i g(Y_j, X_i) - g(X_i, X_j) \\
&= 0
\end{align*}
%
Similarly, $g( \bar{Z}_i, \bar{Z}_j ) = 0$.  
%
We define the symbols of the Riemannian metric $g$ by::
%
$$ g_{ij} = g( Z_i, \bar{Z}_j ) $$
%
then:
%
$$ g( \bar{Z}_i, Z_j ) = \overline{ g( Z_i, \bar{Z}_j ) } = \bar{g}_{ij} $$
%
and we have the following expression for the metric in a holomorphic coordinate frame field:
%
$$ g = g_{ij} d z_i \otimes d \bar{z}_j + \bar{g}_{ij} d \bar{z}_i \otimes d z_j $$

\begin{remark}  It is easy to now derive an expression for the metric in terms of the original real coordinates.  Indeed, it is easy to check that:
%
$$ d z = dx + i dy \quad \text{ and } \quad d \bar{z} = dx - i dy $$
%
so we get:
%
\begin{align*}
g &= g_{ij} dz_i \otimes d \bar{z}_j + \bar{g}_{ij} d \bar{z}_i \otimes d z_j \\
&= g_{ij} ( dx_i + i dy_i ) \otimes ( d x_j - i d y_j ) + \bar{g}_ij (  d x_i - i d y_i  ) \otimes ( dx_j + i dy_j ) \\
%
&= ( g_{ij} + \bar{g}_{ij} ) dx_i \otimes dx_j + ( - i g_{ij} + i \bar{g}_{ij} ) dx_i \otimes dy_j \\
& \quad + ( i g_{ij} - i \bar{g}_{ij} ) dy_i \otimes dx_j + ( g_{ij} + \bar{g}_{ij} ) dy_i \otimes dy_j \\
%
&= 2 \Re (g_{ij}) dx_i \otimes dx_j + 2 \Im( g_{ij} ) dx_i \otimes dy_j \\
& \quad - 2 \Im( g_{ij} ) dy_i \otimes dx_j + 2 \Re ( g_{ij} ) dy_i \otimes dy_j
\end{align*}
%
which gives an expression for the metric in real coordinates.
\end{remark}

Observe that if $g$ is a $J$-invariant metric, then the equation:
%
$$\omega(X,Y) = g(JX, Y)$$
%
defines a real valued 2-form on $M$. Indeed, by the $J$-invariance of $g$, we have:
%
$$ \omega(Y,X) = g(JY, X) = g( J^2 Y, JX ) = - g( Y, JX ) = - g( JX, Y ) = \omega( X, Y ) $$
%
Clearly we can recover $g$ from $\omega$ with the formula:
%
$$ g(X,Y) = \omega(X,JY) $$
%
Again, we also denote by $\omega$ the complex bilinear linear extension of this form to the complexified tangent bundle.  We then have:
%
$$ \omega( Z_i, Z_j ) = g( J Z_i, Z_j ) = i g( Z_i, Z_j ) = 0 $$
%
and similarly $ \omega( \bar{Z}_i, \bar{Z}_j ) = 0 $.  Finally:
%
$$ \omega( Z_i, \bar{Z}_j ) = g( J Z_i, \bar{Z}_j ) = i g( Z_i, \bar{Z}_j ) = i g_{ij} $$
%
so we get the following local expression for $\omega$ in a holomorphic coordinate frame field:
%
$$ \omega = i g_{ij} d z_i \wedge d \bar{z}_j $$
%
In particular, $\omega$ is a $(1,1)$-form. We refer to $\omega$ as the 2-form associated to the metric $g$.

\begin{remark}
While we used a holomorphic coordinate frame field in our calculations, by reviewing the argument above the reader may verify that the results hold in any frame field for the bundle $TM^{+}$.
\end{remark}

\subsection{ The Connection and the Kahler Condition}

Let $\nabla$ denote the Riemannian connection of the metric $g$, and extend it complex linearly in both components to the complexified tangent bundle.  Recall that this connection canonically induces connections on all bundles of tensors associated to $T_{\C}M$, and in particular we can write $\nabla J$ since $J$ is a tensor of type $(1,1)$ on $T_{\C}M$.  As usual, $\nabla_X J$ is again a $(1,1)$ tensor for any vector field $X$, and is defined by the equation:
%
$$ ( \nabla_X J )(Y) = \nabla_X( JY ) - J( \nabla_X Y ) $$
%
With this setup we have the following proposition relating $\nabla J$ and $d \omega$:
%
\begin{prop} Let $M$ be a complex manifold, $g$ a $J$-invariant Riemannian metric, $\omega$ the 2-form associated to $g$, and let $\nabla$ be the Riemannian connection of $g$.  Then we have:
%
$$ d \omega( X, Y, Z ) = g( (\nabla_X J)Y, Z) + g( (\nabla_Y J)Z, X ) + g( (\nabla_Z J)X, Y ) $$
$$ 2 g( (\nabla_X J)Y, Z ) = d \omega(X, Y, Z) - d \omega(X, JY, JZ) $$
%
\end{prop}
%
\begin{proof} We start by verifying the first formula.  It suffices to compute $d \omega( X,Y,Z )$ for $X,Y,Z$ coordinate vectors, and if we use a holomorphic coordinate system, then $X,Y,Z,JX,JY$ and $JZ$ are mutually commuting vector fields (i.e. the Lie bracket of any two of these vector fields vanishes).  Then:
%
\begin{align*}
d \omega( X, Y, Z ) &= X \omega(Y,Z) + Y \omega(Z,X) + Z \omega(X,Y) \\
%
&= X g( JY, Z ) + Y g( JZ, X ) + Z g( JX, Y ) \\
%
&= g( \nabla_X (JY), Z ) + g( JY, \nabla_X Z) + g( \nabla_Y (JZ), X )\\
& \quad + g( JZ, \nabla_Y X ) + g( \nabla_Z (JX), Y ) + g( JX, \nabla_Z Y ) \\
%
&= g( (\nabla_X J)Y, Z ) + g( J( \nabla_X Y ), Z ) + g( JY, \nabla_X Z) \\
& \quad + g( (\nabla_Y J)Z, X ) + g( J( \nabla_Y Z ), X ) + g( JZ, \nabla_Y X ) \\
& \quad + g( (\nabla_Z J)X, Y ) + g( J( \nabla_Z X ), Y ) + g( JX, \nabla_Z Y )
\end{align*}
%
Now we simply observe that, since $[X.Z] = 0$ we have:
%
$$ g( JY, \nabla_X Z) = g( JY, \nabla_Z X ) = - g( Y, J( \nabla_Z X ) )$$
%
and similarly for the terms above ending the two subsequent lines.  In summary, six of the terms in the above expression cancel in pairs, and we are left with:
%
$$ d \omega( X, Y, Z ) = g( (\nabla_X J)Y, Z) + g( (\nabla_Y J)Z, X ) + g( (\nabla_Z J)X, Y ) $$
%
which is what we wanted to show.

To prove the second formula, we write out:
%
\begin{align*}
d \omega( X, JY, JZ ) &= X \omega( JY, JZ ) + JY \omega( JZ, X ) + JZ \omega( X, JY ) \\
&= - X g( Y, JZ ) - JY g( Z, X ) + JZ g( X, Y )
\end{align*}
%
and so the right hand side becomes:
%
\begin{align*}
d \omega( X, Y, Z ) &- d \omega( X, JY, JZ ) \\
&= Y g( JZ, X ) + Z g( JX, Y ) + JY g( Z, X ) - JZ g(X, Z)
\end{align*}
%
Moving on to the left hand side we first write:
%
\begin{align*}
2 g( (\nabla_X J)Y, Z ) &= 2 g( \nabla_X (JY), Z ) - 2 g( J( \nabla_X Y ), Z ) \\
&= 2 g( \nabla_X (JY), Z ) + 2 g( \nabla_X Y, JZ )
\end{align*}
%
and now use the Kozul formula (\cite{CE} page 2).  The results are:
%
\begin{align*}
2 g( \nabla_X (JY), Z ) &= X g( JY, Z ) + JY g( X, Z ) - Z g( X, JY ) \\
&= X g( JY, Z ) + JY g( X, Z ) + Z g( JX, Y ) 
\end{align*}
%
and:
%
\begin{align*}
2 g( \nabla_X Y, JZ ) &= X g( Y, JZ ) + Y g( X, JZ ) - JZ g( X, Y ) \\
&=  - X g( JY, Z ) + Y g( X, JZ ) - JZ g( X, Y )
\end{align*}
%
Adding these gives us:
%
\begin{align*}
2 g( (\nabla_X J)Y, Z ) = Y g( JZ, X ) + Z g( JX, Y ) + JY g( Z, X ) - JZ g(X, Z)
\end{align*}
%
which is what we wanted to show.
%
\end{proof}
%
The next proposition is a cornerstone of Kahler geometry:
%
\begin{prop} Let $M$ be a complex manifold with $g$ a $J$-invariant riemannaian metric, let $\omega$ be the 2-form associated to $g$, and let the $\nabla$ be riemannian connection of $g$.  Then the following conditions are equivalent:
%
\begin{enumerate}
%
\item $ \nabla J = 0 $
%
\item $ d \omega = 0 $
%
\item For every point $p \in M$ there is a smooth real valued function $f$, defined in a neighbourhood of $p$, so that $ \omega = i \del \delbar f$
%
\end{enumerate}
%
\end{prop}
%
\begin{proof}
The previous proposition immediately implies that $1)$ and $2)$ are equivalent. It is also clear that $3) \Rightarrow 2)$, since $d = \del + \delbar$.

We now show that $2) \Rightarrow 3)$.  Assume that $d \omega = 0$.  Now $\omega$ is a real valued 2-form on $TM$, so by the Poincare lemma there is a real valued 1-form $\alpha$, defined in a neighbourhood of any given point $p$, so that $\omega = d \alpha$.  Extending these forms complex linearly to $T_{\C}M$, we can write $\alpha = \beta + \bar{\beta}$, where $\beta$ is a $(1,0)$-form.  Now we have:
%
$$\omega = d \alpha = \del \beta + \delbar \beta + \del \bar{\beta} + \delbar \bar{\beta}$$
%
and since $\omega$ is a $(1,1)$-form, it follows that $\del \beta = \delbar \bar{\beta} = 0$ (since $\del \beta$ is a $(2,0)$-form and $\delbar \bar{\beta}$ is a $(0,2)$-form).  It follows, from the Poincare-Grothendieck lemma, that $\beta = \del \phi$ for some smooth complex function $\phi$ defined near $p$.  Assembling all of this information gives us:
%
$$ \omega = \delbar \beta + \del \bar{\beta} = \delbar \del \phi + \del \delbar \bar{\phi} = \del \delbar( \bar{\phi} - \phi ) $$
%
So setting $f = i( \bar{\phi} - \phi )$ does the trick (note that $f$ so defined is real valued because $\bar{\phi} - \phi$ is imaginary valued).
%
\end{proof}
%
Now we are prepared to make the following important definition:
%
\begin{definition} Let $M$ be a complex manifold with a $J$-invariant riemannian metric.  If any of the three equivelant conditions in the previous proposition are satisfied, then $M$ is called a Kahler manifold, $g$ a Kahler metric, and $\omega$ the Kahler form associated to the metric.  Further, any function $f$ which locally satisfies the condition $3)$ is called a Kahler potential for the metric.
\end{definition}

One of the great advantages of Kahler geometry over Riemannian geometry is the simplification that comes about during computations.  We now investigate how this simplification arises, and derive formulas for the Christoffel symbols of a Kahler manifold.

Let:
%
$$ Z_i = \frac{ \del }{ \del z_i } = \frac{1}{2} \left( \frac{ \del }{ \del x_i } - i \frac{ \del }{ \del y_i } \right) = \frac{1}{2}\left( X_i - i Y_i \right) $$
%
be some local holomorphic coordinate frame field for the bundle $TM^{+}$, with $\bar{Z}_i$ the corresponding frame field for the bundle $TM^{-}$.
Then, using the symmetry of the connection, the fact that all of the relevant brackets vanish, and the fact that $J$ is parallel, we have:
%
\begin{align*}
\nabla_{Z_i} \bar{Z}_j &= \frac{1}{4} \nabla_{(X_i - i J X_i)} (X_j + i J X_j ) \\
&= \frac{1}{4} ( \nabla_{X_i} X_j + i \nabla_{X_i} (J X_j) - i \nabla_{J X_i} X_j + \nabla_{J X_i} ( J X_j ) )\\
&= \frac{1}{4} ( \nabla_{X_i} X_j + i J \nabla_{X_i} X_j - i J \nabla_{X_j} X_i - \nabla_{X_j} X_i ) \\
&= 0
\end{align*}
%
and consequently $\nabla_{\bar{Z}_i} Z_j = \overline{ \nabla_{Z_i} \bar{Z}_j } = 0$ as well.  Finally, we note that, for any complex vector field $X$ on $M$:
%
$$ J \nabla_{X} Z_i = \nabla_{X}(J Z_i) = \nabla_{X}( iZ_i ) = i \nabla_{X} Z_i $$
%
so that $\nabla_{X} Z_j \in TM^{+}$.  Saying this in another way: on a Kahler manifold $M$ the subbundle $TM^{+}$ of $T_{\C} M$ is parallel with respect to the Riemannian connection of a Kahler metric.  In exactly the same way, $TM^{-}$ is also a parallel subbundle of $T_{\C}M$ .

Now an important question must be addressed.  There is a canonical bundle isomorphism $\Phi: TM \rightarrow TM^{+}$ which sends the complex structure $J$ on $TM$ to multiplication by $i$ on $TM^{+}$.  This isomporphism is given by:
%
$$ X \mapsto \frac{1}{2}( X - i JX )$$
%
which we abbreviate as $X \mapsto Z$. Extending the Riemannian connection $\nabla$ on $TM$ to $T_{\C}M$, and then restricting it to the parallel subbundle $TM^{+}$, we get a connection on $TM^{+}$, and it is important for us to be aware of the relationship between these two objects.  Thankfully, the best of all possible results holds:

\begin{prop} Let $X$ and $Y$ be any two vector fields on $TM$, then $\Phi( \nabla_X Y ) = \nabla_{ \Phi X } \Phi Y$.  That is, the connection induced on $TM^{+}$ by the isomorphism $\Phi$ agrees with the one obtained by first extending $\nabla$ to be complex bilinear and then restricting it to $TM^{+}$.
\end{prop}

\begin{proof}
%
It suffices to verify this in a holomorphic coordinate frame field, so let $Z_i = \frac{1}{2}(X_i - i Y_i)$ be such a field. Note that the inverse of the isomorphism $\Phi$ is given by $Z \mapsto Z + \bar{Z}$.  With this in mind, we compute:
%
\begin{align*}
\nabla_{X_i} X_j & = \nabla_{Z_i + \bar{Z}_i} ( Z_j + \bar{Z}_j ) \\
&= \nabla_{Z_i} Z_j + \overline{ \nabla_{Z_i} Z_j }\\
&= \Phi^{-1}( \nabla_{Z_i} Z_j )
\end{align*}
%
so:
%
$$ \Phi ( \nabla_{X_i} X_j ) = \nabla_{Z_i} Z_j = \nabla_{ \Phi X_i } \Phi X_j $$
%
Continuing in this manner:
%
\begin{align*}
\nabla_{Y_i} Y_j &= \nabla_{i(\bar{Z}_i - Z_i)} ( i(\bar{Z}_j - Z_j ) ) \\
&= - (\nabla_{Z_i} Z_j + \overline{ \nabla_{Z_i} Z_j } ) \\
&= - \Phi^{-1}( \nabla_{Z_i} Z_j ) 
\end{align*}
%
so:
%
$$ \Phi( \nabla_{Y_i} Y_j ) = -  \nabla_{Z_i} Z_j =  \nabla_{i Z_i}( i Z_j ) = \nabla_{ \Phi Y_i } \Phi Y_j $$
%
For the mixed terms we use the fact, alluded to earlier, that $\Phi$ sends $J$ to multiplication by $i$.  For:
%
\begin{align*}
\Phi( \nabla_{X_i} Y_j ) &= \Phi( \nabla_{X_i} J X_j ) = i \Phi( \nabla_{X_i} X_j ) = i \nabla_{\Phi X_i} \Phi X_j \\
&= \nabla_{\Phi X_i } i \Phi X_j = \nabla_{\Phi X_i} \Phi Y_j 
\end{align*}
%
The computation involving the other mixed term is entirely analogous. 
%
\end{proof}

Now we investigate the local expression of $\nabla$ on a Kahler manifold.  Again, let $Z_i$ denote and holomorphic coordinate frame field for $M$.  Since $TM^{+}$ is a parallel subundle of $T_{\C}M$, and $\nabla_{\bar{Z}_i} Z_j = 0$, we define the (complex) Christoffel symbols $\Gamma_{ij}^k$ of $\nabla$ by:\footnote{The summation convention on repeated indices will always be in effect}
%
$$ \nabla_{Z_i} Z_j = \Gamma_{ij}^k Z_k $$
%
and observe that we then also have:
%
$$ \nabla_{ \bar{Z}_i } \bar{Z}_j = \bar{\Gamma}_{ij}^k \bar{Z}_k $$
%
We would like to derive an expression for these symbols in terms of the symbols of the Riemannian metric, and this is very easy to do.  Notice that:
%
$$ \Gamma_{ij}^k g_{kl} = g( \nabla_{Z_i} Z_j, \bar{Z}_l ) = Z_i g( Z_j, \bar{Z}_l ) = \frac{ \del g_{jl} }{ \del z_i } $$
%
and therefore:
%
$$ \Gamma_{ij}^k = \Gamma_{ij}^r g_{rl} g^{lk} = \frac{ \del g_{jl} }{ \del z_i } g^{lk} $$
%
This equation expresses the Christoffel symbols (and hence the connection) in a holomorphic coordinate system.  Comparing this to the analogous formula from Riemannian geometry will convince the reader that a great simplification has occurred upon narrowing our focus to Kahler manifolds.

\subsection{The Curvature}

Now let's move on to the Riemannian curvature.  We remind the reader that the Riemannian curvature is a $(3,1)$-tensor that is defined, for $X,Y,Z \in TM$, by:
%
$$ R(X,Y)Z = \nabla_{X} \nabla_{Y} Z - \nabla_{Y} \nabla_{X} Z - \nabla_{[X,Y]} Z $$
%
As has become commonplace, we extend this to be $\C$ bilinear (more honestly, $R$ inherits its complex bi-linearity from $\nabla$).  Then, if $Z_i$ is a holomorphic coordinate frame field for $TM^{+}$, we have (by arguments entirely analogous to those used for $\nabla$):
%
$$ R(Z_i, Z_j) Z_k = 0 $$
%
and:
%
$$ R( \bar{Z}_i, \bar{Z}_j ) \bar{Z}_k = \overline{ R(Z_i, Z_j ) Z_k } = 0 $$
%
Note also that $JR = RJ$, since $J$ is parallel, and hence we have:
%
$$ J R( \bar{Z_i}, Z_j ) Z_k = R( \bar{Z_i}, Z_j )(J Z_k) = R( \bar{Z_i}, Z_j )( i Z_k) = i R( \bar{Z_i}, Z_j ) Z_k $$
%
that is, $ R( \bar{Z_i}, Z_j ) Z_k \in TM^{+} $.  


Of course, $R$ satisfies all of the usual symmetries of a curvature tensor, and in Kahler case additional symmetries appear.  Let $X,Y,Z,W$ be any complex vector fields on a Kahler manifold.  Then:
%
\begin{align*}
g( R(X,Y) JZ, JW ) &=  g( J R(X,Y) Z, JW ) \\
&= g( R(X,Y) Z, W )
\end{align*}
%
and consequently we also have:
%
\begin{align*}
g( R( JX, JY ) Z, W ) &= g( R(Z, W) JX, JY ) \\
&= g( R( Z, W ) X, Y ) \\
&= g( R(X, Y) Z, W )
\end{align*}

Now let's take a look at the local expression for the curvature tensor on a Kahler manifold.  As is the case for the connection, the resulting formulas are drastically more simple than their Riemannian counterparts.  Let $Z_i$ be any holomorphic coordinate frame field for $TM^{+}$, and define the complex curvature symbols by:
%
$$ R( \bar{Z_i}, Z_j ) Z_k = R_{ijk}^l Z_l $$
%
It is easy to derive a local expression for these curvature symbols.  Indeed:
%
\begin{align*}
R( \bar{Z_i}, Z_j ) Z_k &= \nabla_{ \bar{Z}_i } \nabla_{Z_j} Z_k - \nabla_{ Z_j } \nabla_{ \bar{Z}_i } Z_k \\
&= \nabla_{ \bar{Z}_i } \left( \Gamma_{jk}^l Z_l \right ) \\
&= \frac{ \del \Gamma_{jk}^l }{ \del \bar{z}_i } Z_l + \Gamma_{jk}^l \nabla_{ \bar{Z}_i } Z_l \\
&=  \frac{ \del \Gamma_{jk}^l }{ \del \bar{z}_i } Z_l
\end{align*}
%
so we conclude that:
%
$$ R_{ijk}^l = \frac{ \del \Gamma_{jk}^l }{ \del \bar{z}_i } $$


\begin{definition} Let $M$ be a Kahler manifold, and let $X \in TM$. The holomorphic sectional curvature of $M$ at $X$ is defined to be:
$$ K(X) = \frac{ g( R(X, JX) JX, X) }{ g( X, X )^2 } $$
that is, the sectional curvature of the 2-plane spanned by $X$ and $JX$.
\end{definition}
%
In a later section we will need to know how to compute the holomorphic sectional curvature using a holomorphic coordinate frame.  With this in mind we offer:
%
\begin{lem} Given $X \in TM$ write $Z = \frac{1}{2}( X - i JX )$.  Then:
$$ K(X) = \frac{ g( R(X, JX) JX, X) }{ g( X, X )^2 } = - \frac{ g( R(Z, \bar{Z}) \bar{Z}, Z) }{g( Z, \bar{Z} )^2 } $$
\end{lem}
%
\begin{proof} Observe that $X = Z + \bar{Z}$ and $JX = i( \bar{Z} - Z )$.  Now we may simply compute:
%
$$ R(X, JX) = R( Z + \bar{Z}, i( \bar{Z} - Z ) ) = - i( R( \bar{Z}, Z ) - R( Z, \bar{Z} ) ) = - 2i R( \bar{Z}, Z )$$
%
Therefore:
%
$$ g( R(X, JX) JX, X ) = (-2i)^2 g( R(\bar{Z}, Z) Z, \bar{Z} ) = -4 g( R(Z, \bar{Z}) \bar{Z}, Z ) $$
%
On the other hand:
%
$$ g(X, X) = g(Z + \bar{Z}, Z + \bar{Z}) = 2 g(Z, \bar{Z}) $$
%
Putting these two computations together yields the result we are after.
%
\end{proof}
%

\begin{definition} A Kahler manifold is said to have constant holomorphic sectional curvature $c$ when $K(X) = c$ for all $X \in TM$.
\end{definition}
%
In the next section we will give examples of Kahler manifolds with constant holomorphic sectional curvature.  Until then, we offer the reader the following characterization.
%
\begin{prop} A Kahler manifold has constant holomorphic sectional curvature equal to $c$ if and only if its curvature tensor is given by:
%
\begin{align*}
g( R(X,Y)Z, W ) = \frac{c}{4} & ( g(X,W)g(Y,Z) - g(X,Z)g(Y,W) + g(X,JW)g(Y,JZ) \\
& \quad - g(X,JZ)g(Y,JW) + 2g(X,JY)g(W,JZ) )
\end{align*}
%
\end{prop}
%
\begin{proof}
%
Let's use the notation:
%
$$ R(X,Y,Z,W) = g( R(X,Y)Z, W ) $$
%
for the curvature tensor, and write $\Rcurv (X,Y,Z,W)$ for the tensor defined by the right hand side of the equation we are trying to verify.  The reader may directly check that $\Rcurv$ satisfies all of the symmetry properties of the Riemannian curvature tensor\footnote{we call a tensor with these symmetries an algebraic curvature tensor}.  That is:
%
\begin{align*}
\Rcurv (X,Y,Z,W) &= - \Rcurv (Y,X,Z,W) \\
\Rcurv (X,Y,Z,W) &= - \Rcurv (X,Y,W,Z) \\
\Rcurv (X,Y,Z,W) &= \Rcurv(Z,W,X,Y)
\end{align*}
%
and the Bianchi identity holds as well:
$$ \Rcurv (X,Y,Z,W) + \Rcurv (Y,Z,X,W) + \Rcurv (Z,X,Y,W) = 0 $$
%
Furthermore, $\Rcurv$ satisfies the additional Kahler symmetry properties:
%
$$ \Rcurv( JX, JY, Z, W ) = \Rcurv ( X, Y, JZ, JW ) = \Rcurv ( X,Y,Z,W ) $$

Let's now consider the tensor $Q = R - \Rcurv$, which is also an algebraic curvature tensor, and additionally satisfies the Kahler symmetry properties we highlighted earlier.  Upon considering the following calculation:
%
\begin{align*}
\Rcurv (X, JX, JX, X) &= \frac{c}{4} ( g(X,X)g(JX,JX) - g(X,JX)^2 + g(X, JX)^2 \\
& \qquad + g(X,X)g(JX,JX) + 2 g(X,X)^2 ) \\
&= c g(X,X)^2
\end{align*}
%
we see that $M$ has constant holomorphic sectional curvature $c$ if and only if $Q(X,JX,JX,X) = 0$ for all $X \in TM$.  Consequently, to prove the proposition, we need to show that $Q=0$ if and only if $Q(X,JX,JX,X) = 0$ for all $X$.
%
We begin by polarizing the equation $Q(X,JX,JX,X) = 0$:
%
\begin{align*}
0 &= Q ( X + Y, JX + JY, JX + JY, X + Y ) \\
&= 4 Q (X, JX, Y, JX ) +  2 Q (X,Y,X,Y) + 6 Q (X, JY, X, JY) \\
& \qquad + 4 Q ( Y, JY, Y, JX )
\end{align*}
%
Although we have suppressed much of the computation to preserve readability, the symmetry properties of an algebraic curvature tensor and the Kahler symmetry properties were used to simplify the results of the computation above.  Now let's replace $X$ by $tX$ in the equation above, the result is a cubic polynomial in $t$ which must vanish identically, and hence the coefficient of $t^2$ in this polynomial must be zero.  That is:
%
$$2 Q(X,Y,X,Y) + 6 Q(X, JY, X, JY) = 0 $$
%
for all $X,Y \in TM$.  We now apply this last equation twice:
%
$$ Q(X,Y,X,Y) = - 3 Q(X,JY,X,JY) = 9 Q(X, Y, X, Y) $$
%
It follows immediately that $Q(X,Y,Y,X) = 0$ for all $X,Y \in TM$.  

To complete the argument, we cite the well known fact that an algebraic curvature tensor is completely determined by the values of $Q(X,Y,X,Y)$ (any reader who doubts this is invited to consult any textbook on Riemannian gemoetry, for example, \cite{dC}, page 94).  We immediately conclude that $Q = 0$, and hence the argument is completed.
%
\end{proof}

\begin{cor} A Kahler manifold has constant holomorphic sectional curvature if and only if its curvature tensor is given by:
%
\begin{align*}
R(X,Y)Z &= \frac{c}{4} (g(Y,Z)X - g(X,Z)Y - g(Y,JZ)JX  \\
& \quad + g(X, JZ)JY + 2 g(X,JY)JZ )
\end{align*}
%
\end{cor}

\begin{proof}  Simply use the $J$-invariance of $g$ to pass all the $J$'s off of the $W$'s in the formula from the previous proposition.
\end{proof}

The above expression for the curvature tensor in manifolds of constant holomorphic sectional curvature leads to the following important corollary:

\begin{cor} Any two Kahler manifolds with the same constant holomorphic sectional curvature are locally isometric, and if they are both simply connected, then they are globally isometric.
\end{cor}

\begin{proof} This follows immediately from the Cartan-Ambrose-Hicks theorem (see \cite{CE} page 31), using the expression for the curvature tensor in the previous corollary and the fact that $g$ and $J$ are invariant under parallel translation.
\end{proof}

\subsection{ The Ricci Curvature and Kahler-Einstein Manifolds }

We now want to introduce one of the main objects of study in this paper, Kahler-Einstein manifolds.  To begin, we give the important definition:
%
\begin{definition} The Ricci tensor of a Kahler manifold is the complex valued 2-tensor on $M$ defined by:
$$ \Ric (X, Y) = \tr( Z \mapsto R(Z,X)Y )$$
for $X,Y \in T_{\C} M$.
\end{definition}
%
As in the Riemannian case, the Ricci tensor is symmetric.  Therefore, there is an associated $\C$-linear map $T_{\C} M \rightarrow T_{\C} M$, which we also call $\Ric$, defined by:
%
$$ g( \Ric (X), Y ) = \Ric ( X, Y ) $$

The Kahler environment lends itself to a convenient expression for this map.
%
\begin{prop} Suppose that $\{X_i, JX_i\}$, for $i=1,\ldots,n$, is an orthonormal frame for the tangent space $T_P M$.  Then, for $X \in T_p M$, we have:
$$ \Ric (X) = \sum_i R(X_i, JX_i) JX $$
\end{prop}
%
\begin{proof}
We compute, using the usual summation convention to avoid excessive use of $\Sigma$ notation:
%
\begin{align*}
g( \Ric(X), Y ) &= \Ric(X,Y) \\
&= g( R ( X_i, X ) Y, X_i ) + g( R ( J X_i, X ) Y, J X_i ) \\
&= - g( R( X_i, X ) J X_i, J Y ) - g( R( J X_i, X )J X_i, Y ) \\
&= g( R( X, X_i ) J X_i, J Y ) + g( R( J X_i, X ) X_i, JY ) \\
&= - g( R( X_i, J X_i ) X, JY ) \\
&= g( R( X_i, J X_i ) JX, Y )
\end{align*}
%
We have used the Bianchi identity in the fourth line of the computation above. 
%
\end{proof}
%
As a consequence of the above proposition we deduce the $J$-invariance of the Ricci tensor, for:
%
\begin{align*}
\Ric( JX, JY ) &= g( \Ric( JX ), JY ) \\
&= g( R( X_i, J X_i ) J^2 X, JY ) \\
&= g( R(X_i, J X_i ) JX, Y ) \\
&= \Ric( X, Y )
\end{align*}
%
Consequently, we can define an associated $(1,1)$-form, the Ricci form, by:
%
$$ \ric(X,Y) = \Ric(JX, Y)$$

Let's once again investigate how all of this looks in a holomorphic local coordinate system.  Since $\Ric$ is $\C$-bilinear and symmetric, we can repeat the proof we gave for $g$ to show that:
%
$$ \Ric(Z_i, Z_j ) = \Ric( \bar{Z}_i, \bar{Z}_j )= 0 $$
%
So we set:
%
$$ \Ric_{ij} = \Ric( Z_i, \bar{Z}_j ) $$
%
Before we begin deriving a local expression for these symbols, we make a small preliminary calculation.  First note that the trace used in defining the ricci curvature can be calculated by viewing $R$ as a real linear tensor on $TM$ or a complex linear tensor on $T_{\C}M$.  Therefore, if we let $Z_i$ be a local holomorphic coordinate frame field, so that $\{ Z_i, \bar{Z}_i \}$ is a frame field for $T_{\C}M$, and $\{ Z^i, \bar{Z}^i \}$ is a frame field for $T^*_{\C}M$, we have:
%
\begin{align*}
\Ric( Z_i, \bar{Z}_j ) &= \tr( Z \rightarrow R( Z, Z_i ) \bar{Z}_j ) \\
&= Z^k( R(Z_k, Z_i) \bar{Z}_j ) + \bar{Z}^k ( R( \bar{Z}_k, Z_i ) \bar{Z}_j )
\end{align*}
%
Because $R(Z_r, Z_i) = 0$, the first term above is zero.  As for the other, playing around with the curvature tensor gives us:
%
\begin{align*}
R( \bar{Z}_k, Z_i ) \bar{Z}_j &= - \overline{ R( \bar{Z}_i, Z_k ) Z_j } \\
&= - \overline{ R_{ikj}^r Z_r } \\
&= - \bar{R}_{ikj}^r \bar{Z}_r
\end{align*}
%
so $\bar{Z}^k( R( \bar{Z}_k, Z_i ), \bar{Z}_j ) = - \bar{R}^k_{ikj}$.  Therefore:
%
$$ \Ric( Z_i, \bar{Z}_j ) = \tr( Z \rightarrow( Z, Z_i ) \bar{Z}_j ) = - \bar{R}_{ikj}^k = - \frac{ \del \bar{ \Gamma }_{kj}^k }{ \del z_i } $$
%
Setting $G = (g_{ij})$, we have:
%
$$ \Gamma_{kj}^k = \Gamma_{jk}^k = \frac{ \del g_{kl} }{ \del z_j } g^{lk} = \tr \left( \frac{ \del G }{ \del z_j } G^{-1} \right) = \frac{ \del }{ \del z_j } \log \det G $$
%
where in the last equality we have used a classic formula from linear algebra.  Now observe that that, since $G^t = \bar{G}$, we have:
%
$$ \det G = \det G^t = \det \bar{G} = \overline { \det G } $$
%
so $\det G$ is real valued.  Putting this all together, we obtain:
%
$$ \Ric( Z_i, \bar{Z}_j ) = - \frac{ \del \bar{ \Gamma }_{kj}^k }{ \del z_i } = - \frac{ \del }{ \del z_i } \frac{ \del }{ \del \bar{z}_j } \log \det G $$
%
that is:
%
$$ \Ric_{ij} = - \frac{ \del^2 }{ \del z_i \del \bar{z}_j } \log \det G $$
%
This formula becomes particularly simple in the context of the Ricci form, for:
%
$$ \ric = i \Ric_{ij} d z_j \wedge d \bar{z}_i = - i \del \delbar \log \det G $$
%
from which it follows immediately that the Ricci form is a closed $(1,1)$-form.

\begin{definition} A Kahler manifold is called Kahler-Einstein if there is a constant $\lambda$ so that $\ric = \lambda \omega$, or equivalently, $\Ric = \lambda g$.  The constant $\lambda $ is called the Einstein constant of $M$.
\end{definition}

Clearly an Einstein constant is a real number.

\begin{prop} A Kahler manifold, of complex dimension $n$, with constant holomorphic sectional curvature $c$, is a Kahler-Einstein manifold with Einstein constant $\frac{(n+1)c}{2}$.
\end{prop}

\begin{proof}
%
We use the characterization of metrics of constant holomorphic sectional curvature from a previous proposition.  Let $e_1, e_2, \ldots, e_{2n}$ be an orthonormal frame for some tangent space to $M$, and let $X, Y$ be any two tangent vectors at this same point.  Then we have:
%
\begin{align*}
\Ric(X,Y) &= R(X, e_i, e_i, Y) \\
&= \frac{c}{4} ( g(X,Y)g(e_i,e_i) - g(X,e_i)g(Y,e_i) + g(X, JY)g(e_i, Je_i) \\
& \qquad - g(X,Je_i)g(e_i,JY) + 2g(X,Je_i)g(Y,Je_i) ) \\
&= \frac{c}{4}( 2n g(X,Y) - g(X,Y) + g(X,Y) + 2g(X,Y) ) \\
&= \frac{ (n+1) c }{2} g(X,Y)
\end{align*}
%
where we have used that $g(X,e_i)g(e_i,Y) = g(X,Y)$ in the computation.
%
\end{proof}

\end{document}
