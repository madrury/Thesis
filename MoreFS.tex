\documentclass[11pt]{amsart}
\usepackage{epsfig,amsmath}

\newtheorem{thm}[subsection]{Theorem}
\newtheorem{lem}[subsection]{Lemma}
\newtheorem{cor}[subsection]{Corollary}
\newtheorem{prop}[subsection]{Proposition}
\newtheorem{obs}[subsection]{Observation}


\theoremstyle{definition}
\newtheorem{definition}[subsection]{Definition}
\newtheorem{remark}[subsection]{Remark}

\def \CP{ \mathbb{C}P }
\def \C{ \mathbb{C} }
\def \CH{ \mathbb{C}H }
\def \Mamb{ \mathcal{M} }
\def \del{ \partial }
\def \delbar{ \bar{\partial} }
\def \disk{ \mathbb{D} }
\def \blde{ \mathbf{e} }
\def \bldZ{ \mathbf{Z} }


\begin{document}

\parskip 6pt
\parindent 0pt
\baselineskip 14pt

\subsection{ Some More Discussion of the Fubini Study Metric }

We will now give another perspective on the Fubini-Study metric, and study its connection and curvature forms.

Consider the normalization map $z_0 : \C^{n+1} - \{0\} \rightarrow S^{2n-1}$ defined by:
%
$$ z_0(z) = \frac{z}{|z|} $$
%
which is smooth on (since we have removed the origin).  Using the symbol $\cdot$ to denote the usual Hermitian inner product on $\C^{n+1}$, define a hermitian metric on $\C^{n+1} - \{0\}$ by:
%
$$ds^2 = d z_0 \cdot d z_0 - ( d z_0 \cdot z_0 )( z_0 \cdot d z_0 ) $$
%
Observe that this metric is invariant under the multiplicative action of $\C - \{0\}$ on $\C^{n+1}$, and it therefore induces a metric on the quotient space $\CP^{n}$.  We will prove that this is the Fubini-Study metric of constant holomorphic sectional curvature $4$.

Let $p \in \CP^{n}$, and let $Z_0$ be a fixed unit vector in the fiber over $p$.  Then it is easy to see that the projection $\pi : \C^{n+1} - \{0\} \rightarrow \CP^{n}$ induces an isometry from the unitary complement of $Z_0$ equipped with the metric $ds^2$ to the tangent space $T_p \CP^{n}$ with the induced metric, since for vectors in the unitary complement we have $ d z_0 \cdot z_0 = 0 $.  Therefore, given a unitary frame $\blde = \{ e_1, \ldots, e_n \}$ for $T_p \C^n$, there is a unique unitary frame $ \{ Z_1, \ldots, Z_n \} $ for $Z_0^{\perp}$ mapping isometrically to $\blde$.  So suppose we have a local section of unit vectors $Z_0 : U \subset \CP^{n} \rightarrow  \C^{n+1} - \{0\}$ for the projection map, and a local unitary frame field for $TM^{+}$ (where $M = \CP^{n}$) over $U$.  Denoting the $k \times k$ unitary matrices by $U_k$, we can lift all this structure to map:
%
$$ \Phi : U \rightarrow U_{n+1} $$
% 
which is defined in our notation as:
%
$$ \Phi(p) = (Z_0, Z_1, \ldots, Z_n) $$

Now each $d Z_i : U \rightarrow \C^{n+1}$ is a $\C^{n+1}$ valued 1-form, and hence can be expressed as:
%
$$ d Z_i = \phi^j_i Z_j $$
%
where the $\phi$'s are complex valued 1-forms on $U$.  We can write this definition succinctly as $ d \bldZ = \bldZ \phi$, where here we think of $\phi$ as a $M_{n}( \C )$ valued 1-form.  Clearly, these 1-forms can be expressed as:
%
$$ \phi^j_i = d Z_i \cdot Z_j $$
%
Since $d Z_0$ is an isometry from $T_p \CP^n$ to its image in $Z_0^{\perp}$, this immediately implies that:
%
$$ \phi^i_0 = d Z_0 \cdot Z_i = e^i $$
%

Differentiating the equation $ Z_i \cdot Z_j = \delta^i_j $ tells us that:
%
$$ \phi^j_i = - \bar{\phi}^i_j $$
%
that is, the matrix $\phi$ is skew hermitian. 

We can also calculate the exterior derivative of $\phi$.  Let's write:
%
$$ Z_i = ( z_{i0}, z_{i1}, \ldots, z_{in} ) $$
%
then:
%
$$ d Z_i \cdot Z_j = \bar{z}_{jr} d z_{ir} $$
%
and hence, since $\bar{z}_{sr} z_{tr} = \delta^t_s$:
%
\begin{align*}
d( d Z_i \cdot Z_j ) &= d \bar{z}_{jr} \wedge d z_{ir} \\
&= ( \bar{\phi}^s_j \bar{z}_{sr} ) \wedge ( \phi^t_i z_{tr} ) \\
&= \bar{\phi}^s_j \wedge \phi^t_i \bar{z}_{sr}  z_{tr} \\
&= \bar{\phi}^s_j \wedge \phi^s_i \\
&= - \phi^j_s \wedge \phi^s_i
\end{align*}
%
which shows that $d \phi = - \phi \wedge \phi $.   

Next, we relate the forms $\phi$ to the connection forms of the metric $ds^2$ on $\CP^n$.
%
\begin{lem}  Let $ \theta^j_i = \phi^j_i - \delta^j_i \phi^0_0 $, then $\theta$ is the connection form of the metric $ds^2$ with respect to the unitary frame $e_i$.
\end{lem}
%
\begin{proof}  First observe that since $\phi$ is skew hermitian, so is $\theta$.  Therefore we only need to check that the torsion free equation is satisfied:
%
\begin{align*}
d e^i &= d \phi^i_0 = - \phi^i_r \wedge \phi^r_0 \\
&= - \phi^i_r \wedge e^r - \phi^i_0 \wedge \phi^0_0 \\
&= ( - \phi^i_r + \delta^i_r \phi^0_0 ) \wedge e^r \\
&= - \theta^i_r \wedge e^r
\end{align*} 
%
so $d e + \theta \wedge e = 0$, as we wanted to show.
%
\end{proof}
%
\begin{prop} The metric induced on $\CP^n$ by $ds^2$ has constant holomorphic curvature equal to $4$, and hence is isometric to a Fubini-Study metric.
\end{prop}
%
\begin{proof}
%
We calculate the curvature forms using the fundamental equation and the previous lemma:
%
\begin{align*}
\Omega^j_i &= d \theta^j_i + \theta^j_r \wedge \theta^r_i \\
&= d( \phi^j_i - \delta^j_i \phi^0_0 ) + ( \phi^j_r - \delta^j_r \phi^0_0 ) \wedge ( \phi^r_i - \delta^r_i \phi^0_0 ) \\
%
&= - \phi^j_0 \wedge \phi^0_i - \phi^j_r \wedge \phi^r_i + \delta^j_i \phi^0_0 \wedge \phi^0_0 + \delta^j_i \phi^0_r \wedge \phi^r_0 \\
& \quad \phi^j_r \wedge \phi^r_i - \delta^r_i \phi^j_r \wedge \phi^0_0 - \delta^j_r \phi^0_0 \wedge \phi^r_i + \delta^j_r \delta^r_i \phi^0_0 \wedge \phi^0_0 \\
%
&= - \phi^j_0 \wedge \phi^0_i + \delta^j_i \phi^0_r \wedge \phi^r_0 \\
&= e^j \wedge \bar{e}^i - \delta^j_i \bar{e}^r \wedge e^r \\
&= \delta^j_i e^r \wedge \bar{e}^r + e^j \wedge \bar{e}^i 
\end{align*}
%
Now note that since we are using a unitary frame, $h_{ij} = \delta^i_j$, and consequently:
%
$$ \omega = \frac{i}{2} e^r \wedge \bar{e}^r $$
%
and therefore we have:
%
$$ \Omega^j_i = - 2 i \delta^i_j \omega + e^j \wedge \bar{e}^i $$
%
which, from an earlier proposition, implies that the metric has constant holomorphic sectional curvature $4$.
\end{proof}


\end{document}